\part{Theory of RTOS\label{part:rtos-theory}}

\chapter{RTOS characteristics}

\paragraph{}
In order to understand how an RTOS works compared to a general purpose operating system,
    it is essential to define some reccurent characteristics we'd expect to find in a real-time architecture.
\\
This chapter is non exhaustive and one can debate about the importance of one relative to one other.
We decided to chose those characteristics because they are commonly used in the litterature
    and we think that they will provide a good glance at what we can expect from an RTOS.

The first part of this thesis is the result of an effort from ours to summarize the knowledge we collected during our researches.
The following informations do not come from a single source.
We spent countless hours collecting data, reading papers, documentation and technical datasheets from vendors, researchers and developpers.
We also spent time contacting qualified people to review, read and advise us on very specific topics.


\section{System Architecture}

\subsection{Application level}

\subsection{Kernel level}


\section{Sheduling}

\paragraph{}
The scheduler plays an important role in the design of a(n) (RT)OS.
The performances of an operating system are deeply impacted by the scheduling algorithm implemented.
As we'll see further in the thesis, it can also affect the energy consumption of the device.\\

\subsection{Base Concepts}
\paragraph{}
Beforehand, we'll have to explain (or remind) some technical terms in order for the reader to fully understand this chapter.

% What is a scheduler
\paragraph{}
A scheduler is a process designed to choose which task will run at a certain point of time in the system.
Different disciplines can be used, each one with multiple advantages and drawbacks.
Some commonly used will be presented below in the next subsection.

% Preemptive vs non-preemptive (cooperative)
\paragraph{}
A scheduler is said \textit{preemptive} if tasks in the system can preempt each other.
When a higher-priority task wants to execute, the scheduler can interrupt the lower-priority task and run the higher-priority one.
In the other hand, a scheduler is said \textit{non-preemptive} or \textit{cooperative}
    if a task that has been allowed to start will execute until the end.

% online/offline
\paragraph{}
Another distinction we can make is to divide schedulers into \textit{online} and \textit{offline} schedulers.
% online
Online schedulers decide during runtime, based on various parameters (such as task priority for example), the ordering of the different tasks.
A scheduler based on task priority is also called \textit{priority-based} scheduler.
% offline
On the contrary, offline schedulers (also known as \textit{table-driven} schedulers) perform their scheduling decisions at the start of the system.


The impact of these characteristics in the design of an RTOS will be explained later in this chapter.


\subsection{Scheduling Policies}

\subsection{Real-Time Scheduling}
%handbook chapter 2 12
\section{Memory Management}

%https://www.memorymanagement.org/mmref/index.html#mmref-intro
%http://www.csc.twu.ca/rsbook/Ch12/Ch12.4.html
%https://www.gribblelab.org/CBootCamp/7_Memory_Stack_vs_Heap.html
%http://www.cs.virginia.edu/~son/cs414.f05/lec11.slides.pdf




\paragraph{}
In modern computer systems, memory management has evolved since early days techniques which were limited
    by early computer systems where each memory location was specified in the program.
This led to critical errors and/or unpredictability when an incorrect location was specified.

Nowadays, the memory management of (almost) every computer system follows the same principle.
The memory of a computer system is can be divided into 2 distincts sections,
    the static memory or \textit{stack} and the dynamic memory or \textit{heap}.

\subsection{Static Memory Management}
\paragraph{}
%stack
By the time a program begins to execute, there must be some specific blocks of memory set aside for its use.
This includes, for instance, the memory containing the program's own code.
Morever, every static variable must have a specific memory set aside.

The static memory allocation is predetermined by the compiler
    and will always be set aside for the program in the same manner at the beginning of every run.

This part of the memory operates as a \textit{stack} or Last-In First-Out (LIFO) queue.
The area of memory available for the use of the program will shrink and grow following the execution of the program,
which makes it very fast and efficient with no fragmentation.
%explain fragmentation?

\subsection{Dynamic Memory Management}
\paragraph{}
%heap
Sometimes, fixed memory size can be a problem.
Static memory does not allow allocation of memory beyond what is declared initially.
The \textit{heap} serves this purpose.
It is a large pool of memory which must be explicitly managed by the programmer.
It has no guarantee of efficient use of space, memory may become fragmented over time as blocks of memory are reallocated.

It may be tedious for the inexperienced programmer to manage the heap
    but it allows a more flexible and shareable pool of memory for an efficient programming.
To allocate memory on the heap, one must use \textit{malloc()} or \textit{calloc()} from stdlib.h (when available).
There are multiple algorithms to allocate memory when calling these functions.
The most common ones are presented below.

% conventional algorithms
\subsubsection{Sequential Fit Algorithm}
\paragraph{}
For this memory management algorithm, a single linked-list contains the unallocated blocks of memory.
When needed, they are allocated using different policies.
\begin{itemize}
    \item First Fit: returns the first block large enough from the list.
    \item Next Fit: similar to First Fit but starts where the pointer was left off at the previous iteration.
    \item Best Fit: research through the whole list and returns the smallest block large enough to meet the request.
    \item Worst Fit: returns the largest block from the list.
\end{itemize}

\subsubsection{Buddy Allocator Algorithm}
\paragraph{}
This algorithm makes use of an array of linked-lists.
Each list from the array owns blocks from a distinct size.
When requested, the buddy allocator algorithm finds the smallest but large enough block to meet the requirement from the array.
It then picks one of the block from this position in the array.

If the list is empty at the position where the best fitting block is located, it goes to the next position in the array
and splits a block from this list to fill the empty position.
The opposite can be applied too, two smaller blocks can ben merged to obtain a bigger one.

\subsubsection{Indexed Fit Algorithm}
\paragraph{}
This algorithm makes use of an indexed data structure to implement desired fit.
Some common data structures used for this algorithm are trees or hash tables.

\subsubsection{Bitmapped Fit Algorithm}
\paragraph{}
A bitmap representing the usage of the heap is created.
Each bit of the map corresponds to a part of the heap.
If a part is used, the bitmap is set to 1.
If not, it is set to 0.
Allocation is done by searching the bitmap for clear bits.

% unconventional algorithms?

\subsection{Virtual Memory}
\section{Energy management}

%to read
%https://dl.acm.org/citation.cfm?id=2333680
%https://dl.acm.org/citation.cfm?id=860179
%https://dl.acm.org/citation.cfm?id=860184
%https://ieeexplore.ieee.org/abstract/document/4054780
%http://www.es.mdh.se/pdf_publications/327.pdf
%https://www.sciencedirect.com/science/article/pii/S030626191630678X
%https://ieeexplore.ieee.org/abstract/document/5944309
%https://ieeexplore.ieee.org/abstract/document/8617010

%read
%https://www.sciencedirect.com/science/article/pii/S0167739X18329194


% \paragraph{}
% With the rise of cheap battery-powered embedded systems, the problem of energy efficiency becomes a non-negligeable stake.
% Each RTOS has its own way to manage its energy consumption.
% From a more general point of view, we can distinguish 2 ways to save energy in an embedded system.


% \subsection{Hardware energy management}

% At first, one could think that the hardware side of energy management is slightly out of context of the operating system means to save energy.
% The fact is that, even if we can distinguish 2 categories, the hardware means to save energy are very often tied to the upper layers of the whole system.

% \subsection{Software energy management}

\paragraph{}
The focus on energy management is something fairly recent in the field of IT.
With the rise of the Internet of Things, advancements have been made to allow operating systems to manage energy consumption more efficiently.

Numerous communication stacks focused on IoT and low energy consumption have been developped in the last decade.
%cite examples

Unfortunately lesser attention has been paid to the design of energy efficient operating systems for resource-constrained devices.
Traditional hardware is limited in term of energy management and the progress in this field required both software and hardware to evolve.
We will present the advancements made and the technologies developped below.


\subsection{Hardware power management}
\paragraph{}
In order to implement advanced techniques of energy management, certain hardware features have been developped.
The purpose is to give more control from the software over the hardware.

\subsubsection{Clock Gating}
% https://m.eet.com/media/1157354/fpmm%20-%20part%201.pdf
% https://en.wikipedia.org/wiki/Clock_gating
\paragraph{}
Clock Gating is a technique consisting of turning off the clocks of unused peripherals in order to save energy.
Those peripherals enter what is called \textit{idle state} or \textit{sleep}.
The clocks are physically switched off from the circuit with the addition of a logical gate and do not consume energy until reactivation.

\subsubsection{CPU Power Down Modes}
\paragraph{}
The recent advancements in central processing units have introduced power saving modes.
This feature stops the CPU clocks so that it is put on sleep unless
    a scheduling event or interrupt is triggered and wakes the CPU up (with the help of a Real-Time Clock for example).
% How the event can be scheduled if there is no clock to check the event ?

\subsubsection{Real-Time Clock Wake Up Support}
% https://www.electronics-tutorials.ws/connectivity/real-time-clocks.html
\paragraph{}
When a CPU is in sleep mode, there is two possibilities to wake it up from sleep mode: % Link with the previous subsubsection ?
With an on-chip Real-Time Clock (RTC) or by an external event.
The on-chip Real-Time Clock is a low frequency clock (usually around 32kHz) that does not drain a lot of battery life.
RTCs can include alarm functions – timers that when reached trigger the RTC to wake the processor up.

\subsubsection{Supported CPU Frequencies / Dynamic frequency scaling}
\paragraph{}
In modern CPU's, many options are available to switch between frequency ranges depending on the resources needed.
This feature can be used to minimize power consumption when the we don't need much computational power.

\subsubsection{Adaptive Voltage Scaling / Dynamic Voltage Scaling}
\paragraph{}
%https://www.eetimes.com/document.asp?doc_id=1271842#
Similarly to the CPU's frequency, voltage can vary based on the actual state of the chip.
The voltage is continuously monitored and adjusted during the runtime.


\subsection{Operating system power management}

\subsubsection{Peripherals State Control}
\paragraph{}
Peripherals state control makes use of the clock gating feature provided by the hardware.
Thanks to this feature, only the peripherals clocks required by the application at a certain point in time are active.
The other clocks are in halt mode and do not consume energy.

\subsubsection{Sleep Mode}
\paragraph{}
The idea is to allow the system to switch off certain components of the microprocessor.

The sleep mode of a microprocessor takes advantage of multiple hardware features
    such as adaptative voltage scaling, CPU power down modes and dynamic frequency scaling.
The RTC Wake Up Support serves to wake the CPU up when in sleep mode and no other source is active.

\subsubsection{Tick Suppression}
%https://www.embedded.com/electronics-blogs/industry-comment/4414162/1/FreeRTOS-s-tick-suppression-saves-power
\paragraph{}
Tick suppression defines the principle of providing tick-less support for the scheduler.

In a regular scheduler, a periodic timer (tick interrupt) is used to track time.
This tick interrupt wakes the CPU up to perform a scheduling cycle.
Such a mechanism, even if punctual, is frequent and then depletes energy in a non-negligeable way by entering and exiting sleep mode frequently.

In a tick-less scheduler, the tick interrupt is disabled when the idle task is running.
Stopping the tick interrupt allows the CPU to remain in a deep power saving state 
    until either an interrupt occurs or it is time for the kernel to switch task.

%schema tick vs tickless scheduler
\section{Programming Model}
%definition

\subsection{Event-driven model}
In an event-driven programming model, a program generally consists of a main loop which listens for events.
Events can be generated by interrupts, sensors or user input.
When an event is detected, a callback function is triggered.

The developper has to manually maintain state across tasks which can be tedious.
Thus, the individual tasks do not have to maintain their own stack and they use a shared-stack.
The memory footprint is then reduced since a single stack is used across the application.

\subsection{Multithreaded model}
The multithreaded model allows an application to run different tasks in their own thread context,
    and communicate between them using an Inter-Process Communication API.

Each thread has its own memory stack and does not require manual management by the programmer.
The stack is managed automatically by the thread scheduler.
The memory requirements of threaded application are often larger than their event-driven counterparts.
\section{Hardware Support -- DRAFT}

\subsection{Constrained devices classes}

Constrained devices have been classified in 3 classes by the IETF with RFC7228 in May 2014. The distinction between those three classes are made with the RAM and ROM capabilities.

\begin{table}[!h]
  \centering
  \begin{tabular}{|l|l|l|}
  \hline
   & data size (e.g., RAM) & code size (e.g., Flash) \\ \hline
  Class 0, C0 & \textless{}\textless 10 KiB & \textless{}\textless 100 KiB \\ %\hline
  Class 1, C1 & $\sim$10 Kib & $\sim$10 KiB \\ %\hline
  Class 2, C2 & $\sim$25 KiB & $\sim$250 KiB \\ \hline
  \end{tabular}
  \caption{Classes of Constrained Devices}
  \label{constrained-devices-classes}
\end{table}

\paragraph{Class 0}
Those devices are the most constrained. There are typically sensor-like motes. There are so constrained that they cannot access Internet without the help of a larger devices. From the point of view of a RTOS, their code are too heavy to fit in such devices. Instead Class-0 devices are usually used bare metal.

\paragraph{Class 1}
Those devices are able to talk to each other but via constrained protocols. Use of security protocols are too heavy for that class. RTOS are mainly focused on these kind of devices.

\paragraph{Class 2}
Those devices are the less constrained and can use the same stacks of protocols used in personal computers and servers. General-purpose operating systems can be used for these kind of devices but the Class-2 devices can benefit from lightweight and energy-efficient protocols.

\subsection{Hardware Abstraction Layer}
Some OS are tied to a specific hardware architecture and target only specific vendors making the OS not portable. Developers cannot change the hardware during the development process. Developers cannot reuse their code with other platform. 

\paragraph{Definition of HAL}
An hardware abstraction layer defines a set of routines, protocols and tools to access underlying hardware. It provides abstract and high-level functions to interact with the hardware. The hardware, drives and board supports are considered as a black-box.


\paragraph{CPU diversity}
The embedded world use the majority for the world's microcontrollers. The diversity of CPU keep increasing every year. 
In RIOT, the CPU are organized in an hierarchical way. The hardware abstraction layer is split among the CPU architecture (e.g. Cortex M-4), the CPU family (e.g. stm32), the CPU type (e.g. stm32f4) and, finally, the CPU model (e.g. stm32f401re).
Using this kind of structure help to avoid duplication of code.

\paragraph{Pro and cons}
Developers can switch hardware and perform cross-platform testing more easily. But there is some limitation. The hardware abstraction layer is tied to the hardware and change heavily with the hardware. Also, there is some limitation using an hardware abstraction layer. Not all the functionnalities from the hardware are available. 


\chapter{RTOS Outline}

\paragraph{}
In this chapter, we will cover the specificities of each operating system we chose to work with.
We chose to write about technical and non-technical characteristics of the chosen OS's and even their respectives projects as a whole.

Some of the things written in this chapter are personal opinions.
Indeed, as we decided to include non-technical characteristics of each of the projects, some subjectivity is involved.
We tried to be critical and impartial in those points but it beware it only reflects our own opinions.


A table summarizing the comparison is provided at the end of the chapter.

%https://www.simform.com/iot-rtos-selection/
%https://ijarcce.com/wp-content/uploads/2015/12/IJARCCE-92.pdf
%http://www.ipcsit.com/vol1/54-S028.pdf

%Supported Architecture
%License
%User-friendly/documentation/example applications
%Supported network protocols
%kernel type
%date of creation
%file systems supported
%number of boards supported
%memory footprint
%resource access control (posix...)
%memory management policy
%energy management features
%programming model
%scheduling
%community
%build system
%ipc management

\section{RIOT OS}

\subsection{Historic}
\paragraph{}
The RIOT project started privately in 2008.
It started as a part of the FeuerWhere project (\url{http://feuerwhere.mi.fu-berlin.de}), where firefighters would be tracked with embedded devices during an intervention.
The goal was to design an ad-hoc self configurating network of sensors used to monitor vital state and environement parameters of rescue forces inside a building.

% https://ieeexplore.ieee.org/stamp/stamp.jsp?tp=&arnumber=6142316
In 2010, a fork from the FeuerWare software developped for the FeuerWhere project was made.
Developpement continued for $\mu$kleos, a microkernel based operating system for embedded devices.
The focus of this system was modularity and Internet compliance with the integration of IETF protocols such as 6LoWPAN, RPL and TCP.

In 2013, RIOT went public.
$\mu$kleos was rebranded to avoid problems with spelling and pronounciation.
Since then, more than 200 people contributed to the project with more than 20 000 commits and 75 releases.
The project is currently supported by Freie Universität Berlin, INRIA and Hamburg University of Applied Sciences.

\subsection{Characteristics and features}
%ce que fait l'os

\paragraph{}


\subsection{Specificities}
%tickless
\section{Contiki}
%explain contiki and why we chose it

\subsection{Historic}
\paragraph{}
%https://ieeexplore.ieee.org/stamp/stamp.jsp?tp=&arnumber=1367266
%https://ercim-news.ercim.eu/en76/rd/contiki-bringing-ip-to-sensor-networks
Contiki was created in 2003 by Adam Dunkels. %http://dunkels.com/adam/
At the time, it was the first operating system to provide IP connectivity to sensor networks.
It did so thanks to $\mu$IP, a lightweight TCP/IP stack intended for tiny microcontrollers,
    also developped by Adam Dunkels at the Swedish Institute of Computer Science (now RISE SICS).%https://github.com/adamdunkels/uip

Further developpement of Contiki has been supported by various industries and research institutes 
    such as Texas Instruments, Atmel, Cisco, ENEA, ETH Zurich, Redwire, RWTH Aachen University, 
    Oxford University, SAP, Sensinode, Swedish Institute of Computer Science, ST Microelectronics and Zolertia.
%death?

\subsection{Characteristics and features}
\paragraph{}
%protothreads

%memory allocation/stack and kernel/event driven

%dynamic loading

\subsection{Specificities}
\paragraph{}
%contiki is specific by itself, aims for WSN
%cooja
%rime https://pdfs.semanticscholar.org/9feb/7e0f0d3b507f2f3da60c1b2fea9d5e43449d.pdf
%protothread concept?
\section{Comparison table}

The table \ref{tab:rtos-comparison} summarize the comparison between RIOT, Contiki and FreeRTOS.

\begin{table}[!ht]
  \centering
  \small
  \setlength\tabcolsep{1pt}
  \begin{tabular}{l|c|c|c|c|c|c}
           & RAM size       & ROM size        & Scheduler type & Programming model &  License     & Communities \\ \hline
  RIOT     & $\sim$1.5kB    & $\sim$5kB       & preemptive     & mutli-thread      &  LGPL2.1     & ++          \\
  Contiki  & \textless{}2kB & \textless{}30kB & cooperative    & event-driven      & Revised BSD & +           \\
  FreeRTOS & \textless{}1kB & \textless{}10kB & configurable   & mutli-thread      & MIT         & -          
  \end{tabular}
  \caption{RTOS comparisons}
  \label{tab:rtos-comparison}
  \end{table}