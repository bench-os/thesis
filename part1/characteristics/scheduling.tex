\section{Sheduling}

\paragraph{}
The scheduler plays an important role in the design of a(n) (RT)OS.
The performances of an operating system are deeply impacted by the scheduling algorithm implemented.
As we'll see further in the thesis, it can also affect the energy consumption of the device.\\

\subsection{Base Concepts}

% What is a scheduler
\paragraph{}
A scheduler is a process designed to choose which task will run at a certain point of time in the system.
Different disciplines can be used, each one with multiple advantages and drawbacks.
Some commonly used will be presented below in the next subsection.

% Preemptive vs non-preemptive (cooperative)
\paragraph{}
A scheduler is said \textit{preemptive} if tasks in the system can preempt each other.
When a higher-priority task wants to execute, the scheduler can interrupt the lower-priority task and run the higher-priority one.
In the other hand, a scheduler is said \textit{non-preemptive} or \textit{cooperative}
    if a task that has been allowed to start will execute until the end.

% online/offline
\paragraph{}
Another distinction we can make is to divide schedulers into \textit{online} and \textit{offline} schedulers.
% online
Online schedulers decide during runtime, based on various parameters (such as task priority for example), the ordering of the different tasks.
A scheduler based on task priority is also called \textit{priority-based} scheduler.
% offline
On the contrary, offline schedulers (also known as \textit{table-driven} schedulers) perform their scheduling decisions at the start of the system.

The impact of these characteristics in the design of an RTOS will be explained later in this chapter.

\subsection{Scheduling Policies}

\subsubsection{First in, first out (FIFO)}
Also known as First come, first serve (FCFS), FIFO is one of the simpliest scheduling algorithm.
Processes are queued into a data structure in their order of arrival.
The first process to be enqueued is the first which will be executed.

\subsubsection{Round-robin}
The round-robin scheduling algorithm divides the time allocated to each process into fixed time slices.
If a process doesn't terminate when his allocated time slice is expired, it is preempted and the scheduler switches to another task.
If a task terminates within its time slice, the scheduler simply switches to the next task.

\subsubsection{Earliest deadline first (EDF)}
Earliest deadline first scheduling algorithm dynamically assigns a priority
    to each enqueued process (based on its deadline or an estimation of it) into a priority queue data structure.
The scheduler then executes the process the closest to its deadline at each scheduling event.

\subsubsection{Fixed priority preemptive scheduling}
Priority for each task is pre-assigned by the operating system.
The scheduler arranges the tasks by order of priority.
Higher priority tasks can interrupt lower priority tasks.

\subsection{Real-Time Scheduling}
%handbook chapter 2 12
