\section{RIOT OS}

\subsection{Historic}
\paragraph{}
The RIOT project started privately in 2008.
It started as a part of the FeuerWhere project (\url{http://feuerwhere.mi.fu-berlin.de}), where firefighters would be tracked with embedded devices during an intervention.
The goal was to design an ad-hoc self configurating network of sensors used to monitor vital state and environement parameters of rescue forces inside a building.

% https://ieeexplore.ieee.org/stamp/stamp.jsp?tp=&arnumber=6142316
In 2010, a fork from the FeuerWare software developped for the FeuerWhere project was made.
Developpement continued for $\mu$kleos, a microkernel based operating system for embedded devices.
The focus of this system was modularity and Internet compliance with the integration of IETF protocols such as 6LoWPAN, RPL and TCP.

In 2013, RIOT went public.
$\mu$kleos was rebranded to avoid problems with spelling and pronounciation.
Since then, more than 200 people contributed to the project with more than 20 000 commits and 75 releases.
The project is currently supported by Freie Universität Berlin, INRIA and Hamburg University of Applied Sciences.

\subsection{Characteristics and features}
%ce que fait l'os

\paragraph{}


\subsection{Specificities}
%tickless