\section{Contiki}
%explain contiki and why we chose it

\subsection{Historic}
\paragraph{}
%https://ieeexplore.ieee.org/stamp/stamp.jsp?tp=&arnumber=1367266
%https://ercim-news.ercim.eu/en76/rd/contiki-bringing-ip-to-sensor-networks
Contiki was created in 2003 by Adam Dunkels. %http://dunkels.com/adam/
At the time, it was the first operating system to provide IP connectivity to sensor networks.
It did so thanks to $\mu$IP, a lightweight TCP/IP stack intended for tiny microcontrollers,
    also developped by Adam Dunkels at the Swedish Institute of Computer Science (now RISE SICS).%https://github.com/adamdunkels/uip

Further developpement of Contiki has been supported by various industries and research institutes 
    such as Texas Instruments, Atmel, Cisco, ENEA, ETH Zurich, Redwire, RWTH Aachen University, 
    Oxford University, SAP, Sensinode, Swedish Institute of Computer Science, ST Microelectronics and Zolertia.
%death?

\subsection{Characteristics and features}
\paragraph{}
%protothreads

%memory allocation/stack and kernel/event driven

%dynamic loading

\subsection{Specificities}
\paragraph{}
%contiki is specific by itself, aims for WSN
%cooja
%rime https://pdfs.semanticscholar.org/9feb/7e0f0d3b507f2f3da60c1b2fea9d5e43449d.pdf
%protothread concept?