\section{Benchmarking Framework}
 
As we were not pleased by the previous objective, we decided to change it.
We wanted to build a framework hidden for the developer that he or she could run on any RTOS with any application.

The framework would be able to retrieve measurements such as the context switching time, the interrupt latency, the memory usage and the energy consumption.

An use case for our benchmarking framework is explain in the section below.

\subsection{Use case}
Ideally, using our benchmarking framework, the use case for any developer using a RTOS would be the following.

The developer develops an application with multiple tasks on a specific RTOS on a specific device.
The developer wants to measure the performance of its application and, to do so, set a flag in the Makefile of its application to turn on the benchmarking framework.
The developer flash its application on the device and the framework output continuously benchmarking information.
The information contains context switching time, interrupt latency, memory usage and enery consumption.

With this framework, the developer can optimize its application, change the device or even the RTOS.
The framework, available on as much RTOS as possible, will help the developer to see which RTOS or devices is the best fit for its application.

\subsection{Limitations}
As the next chapter explains, this framework idea encountered several limitations.
First, the devices does not have enough internal power to compute the context switching time or the interrupt latency with engough precision.
Second, we did not had enough time to build a framework able to measure the memory usage or the energy consumption.