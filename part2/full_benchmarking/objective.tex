\section{Objective}

Our first objective for this thesis was to benchmark open-source RTOS such as RIOT-OS, Contiki or FreeRTOS.
The benchmark would be based on a set of criteria defined later in this chapter.
With this benchmark, we would have been able to determine which RTOS performs better for specifics tasks.

We also went to the RIOT Summit 2018 in order to meet the community behind the RIOT-OS.
Communicate with the RTOS community is an important step in our thesis.
This allow us to better understand the use of RTOS as well as the need for the developers.

\subsection{Criteria}
We have defined a set of criteria to assess the performance of each RTOS.
This list is not exhaustive and more information about each criterion can be found in the part \ref{part:rtos-theory} about the theory of RTOS.

\paragraph{Scheduling}
The scheduling is a major piece of any RTOS and, as such, must be benchmarked.

\paragraph{Context Switching}
The time taken to switch from a task to anoher must be the lowest possible.

\paragraph{Interrupt Latency}
Related to the context switching, the interrupt latency is the time between an interrupt occurs and the time it is handled.
The lower the better.

\paragraph{Memory Usage}
Constrained devices do not have a lot a memory.
For this reason, RTOS should use the memory in an efficient way.

\paragraph{Energy Consumption}
Many RTOS optimize their code in such a way that the energy consumption is the lowest.

\paragraph{Networking Stacks}
In the context of the Internet of Things, communicating with other devices is a must-have feature.

\subsection{RIOT Summit 2018}
We went to the RIOT Summit 2018 in Amsterdam in order to meet developers and researchers specialized in the RTOS area.

This was the third summit of the RIOT community.
Every year the members of this community gather and talk about their projects and the future work for RIOT-OS.
The summit was divided into two days. During the first day, 12 speakers presented their work. On the second day, tutorials were given and breakout groups ended the summit.

By talking to developers that were present at the summit, we learned that the STM32F4 series microcontroller is a good choice to perform a benchmark on RTOS. With its ARM Cortex-M4 based MCU, it compiles a large variety of RTOS.

Additionaly, we talked about our idea to use a logical analyser to perform time analysis. Gilles Doffe from Savoir-faire Linux confirmed our opinion about using this kind of devices for our benchmarking.

During this summit, we discovered a large number of application of RIOT-OS and RTOS in general.
We talked to some of the maintainers of RIOT-OS and other developers. With their expertise and their advices, we got references to hardwares and softwares that could be useful for our work.

\subsection{Objective shift}

After discussing with the RTOS community and after reading papers about benchmarking of RTOS, we decided to shift our objective.
Making a full benchmark of RTOS left us with two problems.

First, many works and researches have already been done on this subject.
Lot of papers assess the performance of RTOS such as RIOT, Contiki or FreeRTOS comparing them.

Then, the need to define specific tasks in order to benchmark a RTOS does not allow a developer to evaluate the real performance of its application on a RTOS.

For those two reasons, we decided to change our objective from performing a full benchmark to develop a benchmarking framework.
This new objective is described in the next section.