\section{Results}

The results are split in two parts.
The first part is the context switching time value measured by our benchmarking framework.
For the second part, we have measured again the real context switching time with the oscilloscope the same way as for the reference measurement.

\paragraph{Context switching time measured by our framework} 
The value measured by our framework is represented in the table \ref{tab:framework-measurement}.

\begin{table}[!h]
  \centering
  \begin{tabular}{llll}
                        & \multicolumn{3}{c}{Time ($\mu$s)}          \\ \cline{2-4} 
                        & \multicolumn{1}{c}{Mean} & Min  & Max  \\ \cline{2-4} 
  From task 1 to task 2 & 7812                     & 7812 & 7812 \\
  From task 2 to task 1 & 7812                     & 7812 & 7812
  \end{tabular}
  \caption{Context switching time measured by our benchmarking framework}
  \label{tab:framework-measurement}
  \end{table}

\paragraph{Context switching time measured by the oscilloscope}
Using the same setup as for the reference measurement, we have measured again the real context switching time.
The figure \ref{fig:framework-value-wave} shows the voltage measured by the oscilloscope while using our framework.
The table \ref{tab:framework-measurement} shows the duration of the two context switchings and the two tasks.

\begin{figure}[!ht]
  \centering
  \includegraphics[scale=0.5]{assets/framework-value-wave.png}
  \caption{\label{fig:framework-value-wave}Voltage measurement of the two GPIOs; Each color represents a task}
\end{figure}

\begin{table}[!ht]
  \centering
  \begin{tabular}{llll}
                        & \multicolumn{3}{c}{Time ($\mu$s)}                             \\ \cline{2-4} 
                        & \multicolumn{1}{c}{Mean} & Min  & \multicolumn{1}{c}{Max} \\ \cline{2-4} 
  From task 1 to task 2 & 1864                     & 1863 & 1866                    \\
  From task 2 to task 1 & 1865                     & 1862 & 1866                    \\
  Duration of task 1    & 1003                     & 1003 & 1003                    \\
  Duration of task 2    & 1003                     & 1003 & 1003                   
  \end{tabular}
  \caption{Context switching times and task durations measured with the oscilloscope Tektronix MSO 56 using our framework}
  \label{tab:framework-measurement}
\end{table}

\subsection{Discussion}

\paragraph{Comparison with the reference measurement}

\begin{table}[!ht]
  \centering
  \begin{tabular}{lll}
                        & \multicolumn{2}{c}{Time ($\mu$s)}                                     \\ \cline{2-3} 
                        & \multicolumn{1}{c}{Without the framework} & With the framework \\ \cline{2-3} 
  From task 1 to task 2 & 14.68                                     & 1864                  \\
  From task 2 to task 1 & 14.88                                     & 1865                  \\
  Duration of task 1    & 1003                                      & 1003                  \\
  Duration of task 2    & 1003                                      & 1003                 
  \end{tabular}
  \caption{Comparison}
  \label{tab:measurements-comparison}
\end{table}

\paragraph{Limitations of the framework}

% As the next chapter explains, this framework idea encountered several limitations.
% First, the devices does not have enough internal power to compute the context switching time or the interrupt latency with engough precision.
% Second, we did not had enough time to build a framework able to measure the memory usage or the energy consumption.
