\section{Implementing a benchmarking framework}

Assessing the performance of an RTOS is a primordial step during the development of an application in the embedded world.
For example, computing the interrupt latency or the context switching time requires using an oscilloscope.
However, this kind of devices is costly and some developers can not afford them.
Furthermore, working with external devices takes time and can dramatically slow down the development process.

For those reasons, we wanted to build a benchmarking framework hidden from the developer that he could run on any RTOS with any application.
Ideally, this framework would respect the following criteria in order to be useful for a real world application.

\paragraph{Widely distributed}
By making our framework accessible on a large variety of RTOS, we allow developers not to limit themself to a specific set of RTOS.
With this framework, they would be able to assess the performance of their applications regardless of the RTOS.

\paragraph{Providing a large variety of metrics}
Our framework should be able to provide performance measurements on interrupt latency, context switching time.
Those metrics are important for a real-time context applications.
Also, using constrained devices requires to monitor the memory usage and the energy consumption of the applications.
Our framework should be able to provide those metrics as well.

\paragraph{Hidden for the developer}
The developer should not need to worry about integrating our framework in its application.
Using or not our framework should not change the source code of the application and allow the developer to directly deploy it in a production environment.

\paragraph{Easy to use and to configure}
Using an oscilloscope or any external device to benchmark an embedded application takes time and effort for the developer.
Our framework should be able to improve the benchmarking process by being easy to use and easy to configure.

\subsection{Narrowed objective}

In the context of our thesis, we choose to focus our work on a restricted objective.
We have decided to produce a proof-of-concept showing that it is possible to achieve our biggest objective given more time.
For our proof-of-concept, we have reduce our framework implementation to a benchmarking of the context switching time on the Contiki RTOS.

We chose to use Contiki for its cooperative scheduler.
This kind of scheduler allows us to know when a task is in the foreground and when it goes to the background.
With a preemptive scheduler, in the other hand, we do not know when the task is preempted.
This issue is discussed later in our work.
In Contiki, the developer can use macros like \texttt{PROCESS\_PAUSE()} or \texttt{PROCESS\_WAIT\_EVENT()} to let other tasks run in a cooperative fashion.
% https://github.com/contiki-os/contiki/wiki/Processes
The complete list of macros and more information can be found in the \href{https://github.com/contiki-os/contiki/wiki/Processes}{Contiki processes documentation}.

We choose to focus on the context switching time for the metric of our benchmarking framework because, in the context of embedded devices, the time spent between two tasks must be the smallest possible.
Benchmarking the context switching time can help the developer to reduce the time lost between two tasks.
Moreover, by considering the interrupt handler as a task, it is possible to compute the interrupt latency with the context switching time.

From the criteria previously described, we have decided to reduce the first two by focusing only on Contiki and the context switching time.
For the last two, we leave them untouched in order to build a benchmarking framework the most hidden and easy-to-use for the developer.

We left the goal to implement a framework capable of benchmarking the interrupt latency, the memory usage and the energy consumption of any RTOS to future work.

\subsubsection{Use case}
Ideally, the use case for any developer using our benchmarking framework would be the following.
This scenario is hypothetic and does not represent the state of our work.

The developer implements an application with multiple tasks on a Contiki on a specific device.
The developer wants to measure the performance of its application and, to do so, set a flag in the Makefile of its application to turn on the benchmarking framework.
The developer flashes its application on the device and the framework output continuously benchmarking information including the context switching time between the tasks of the application.

With this framework, the developer can optimize its application or change the device and check if the context switching time is improved.

\subsubsection{Simple task}
In order to experiment and test our benchmarking framework, we need to have an application running on Contiki.
For this reason, we implemented a simple application.

The application contains two indentical tasks.
The tasks wait for 1ms in the foreground and then goes to the background letting the scheduler runs the next task.
The tasks run in an infinite loop.
The source code of one of the two tasks is shown in the listing \ref{lst:simple-task-code}.

\begin{lstlisting}[style=CStyle, label={lst:simple-task-code}, caption={Source code of a task implemented in Contiki for the simple application}]
PROCESS_THREAD(task, ev, data)
{
    PROCESS_BEGIN();

    while (1)
    {
        PROCESS_PAUSE();
        // Wait for 1ms
        clock_delay_usec(1000);
    }

    PROCESS_END();
}
\end{lstlisting}