\section{Discussions}

\subsection{Internal benchmarking framework}

By comparing with our reference measurement, the first assessment we can make is that our benchmarking framework does not compute the context switching time correctly.
The framework measure a context switching time of 7812 $\mu$s where we expect a value of 14.68 $\mu$s.

Second assessment we can make is that our framework add a huge overhead.
When comparing the figure \ref{fig:measurement-value-wave} with the figure \ref{fig:internal-framework-value-wave}, the overhead is largely visible.
By comparing the values measured by the oscilloscope, the reference measurement and the real context switching time while using our framework, we see that our framework add an overhead of 1850 $\mu$s.
The table \ref{tab:measurements-comparison} shows this comparison.

\begin{table}[!ht]
  \centering
  \begin{tabular}{lll}
                        & \multicolumn{2}{c}{Time ($\mu$s)}                                     \\ \cline{2-3} 
                        & \multicolumn{1}{c}{No framework} & Framework \\ \cline{2-3} 
  From task 1 to task 2 & 14.68                                     & 1864                  \\
  From task 2 to task 1 & 14.88                                     & 1865                  \\
  Duration of task 1    & 1003                                      & 1003                  \\
  Duration of task 2    & 1003                                      & 1003                 
  \end{tabular}
  \caption{Comparison of the oscilloscope measurements}
  \label{tab:measurements-comparison}
\end{table}

\subsubsection{Limitations of the framework}

As the results show, our framework suffers from several limitations.

First, the devices can not compute the context switching time with enough precision.
This lack of precision is due of the speed of the device clock that is not high enough.
In our experiment, the Zolertia RE-MOTE used with Contiki could only use a timer with 128 ticks per second or 1 tick every 7812 $\mu$s.

Second, our framework output its computed value through serial port.
Printing out at least 32 characters for every context switch on the serial port even at 250 kbit/s took 1.2ms.
One optimization could be to reduce the number of bits send or use a cache.

\subsection{External benchmarking framework}

The first thing we can see is that our external benchmarking framework is much more precise that our internal benchmarking framework.
Even if the result does not match our excpectation, the results show an improvement.
We went from a measurement error with a factor of more than 500 to a measurement error with a factor less than 2.

Secondly, with the real context switching time measured at 14.87$\mu$s while our framework was on, our external benchmarking framework does not add an overhead.

In conclusion, our external benchmarking framework is more precise than our internal benchmarking framework and does not add an overhead to the real context switching time.