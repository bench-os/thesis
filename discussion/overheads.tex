\section{Extension approach overhead issue\label{sec:overhead}}

With our first approach, the framework adds to our simple application a large overhead of more than 2ms for every context switch.
This is an issue for our benchmarking framework as it should have a minimal impact on our application.
With such an overhead, the RTOS lose its real-time capabilities.
It becomes impossible for the RTOS to react in time to interrupts.

\subsection{Causes}

Sending measurements through the serial port create this latency.
Indeed, the framework sends a string that contains the measured value in order for the computer to read it.
For example, the listing \ref{lst:print-string} shows how the framework write the measurements on the serial port with the method \texttt{printf()} with Contiki

\begin{lstlisting}[style=CStyle, float, label={lst:print-string}, caption={writting the measurements to the serial port with Contiki}]
  printf("[BENCH_CONTEXT_SWITCHING] %lu %lu %lu\n", previous_id, bench_context.new_id, result);
\end{lstlisting}

The string \texttt{[BENCH\_CONTEXT\_SWITCHING]} is a flag used to differentiate framework outputs from the application outputs.
However, printing out at least 32 characters in every context switch on the serial port even at 250 kbit/s took 1.2ms.

\subsection{Solutions and improvements}

To improve the extension approach, one optimization could be to reduce the number of bits send through the serial port.
By sending compressed data and using only a small number of bytes, it could be possible to reduce the overhead.
But even with 8 bits for the flag a 32 bits for the context switching time, 40 bits would take up to $160\mu s$ to be transmitted.

In addition to compress the data, another solution would be to use a cache storing the measurements.
This cache would be flushed and the values written on the serial port at appropriate times.
Either when the cache is full or with a framework method that would allow the developer to choose when to flush the cache.
This way, the overhead occurs not at each context switch but only at an appropriate time.

\subsection{Overhead of the devices approach}

On the other hand, the devices approach have a small impact on the system with an overhead of less than $3\mu s$ per context switch.
Even if it is a small overhead, it is however present.
Our hypothesis on this overhead is that the framework calls, \texttt{bench\_on()} and \texttt{bench\_off()} have an impact on the system.
However, there is no fact supporting this conclusion.