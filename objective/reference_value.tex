\section{Reference value \label{sec:reference}}

Our benchmarking framework will compute the context switching time but we need to know how well or how bad our framework performs.
To assess the performance of our measurements, we first need to compute the real context switching time.
This real context switching time will be our reference value.

\subsection{Simple application}
In order to experiment and test our benchmarking framework, we need to have a simple application for every tested RTOS.

The application contains two identical tasks.
The tasks wait for 1 ms in the foreground and then goes to the background letting the scheduler run the next task.
The tasks run in an infinite loop.

\subsubsection{Implementation in Contiki}
The source code of one of the two tasks is shown in the listing \ref{lst:simple-task-code}.
It uses the \texttt{clock\_delay\_usec()} call to make the task wait for 1 ms.
Finally, we use the \texttt{PROCESS\_PAUSE()} macro to pause the task and let the other task do its iteration.

\begin{lstlisting}[style=CStyle, float, label={lst:simple-task-code}, caption={source code of a task implemented in Contiki for the simple application}]
PROCESS_THREAD(task, ev, data)
{
    PROCESS_BEGIN();

    while (1)
    {
        clock_delay_usec(1000);
        PROCESS_PAUSE();
    }

    PROCESS_END();
}
\end{lstlisting}

\subsubsection{Implementation in RIOT}
The source code of on of the two tasks is shown in the listing \ref{lst:simple-task-code-riot}.
We cannot use the \texttt{xtimer\_usleep()} method as it will lead to a context switch.
Instead, we used a for loop.
The \texttt{thread\_yield()} call will perform a context switch and let the other task run on iteration.

\begin{lstlisting}[style=CStyle, float, label={lst:simple-task-code-riot}, caption={source code of a task implemented in RIOT for the simple application}]
void *threadA(void *arg)
{
    (void)arg;

    while (1)
    {
        for(int i = 0; i < 1000; i++) {}
        thread_yield();
    }
    return NULL;
}
\end{lstlisting}


\subsection{Methodology}

In order to compute the real context switching time, we used an oscilloscope and two GPIOs.
Each task of our simple application is responsible of one GPIO.
In order for the task to use the GPIO, they need to be updated.
We describe this change in the subsection \ref{sec:measurement-setup}.
Every time a task is run and goes to the foreground, it sets its GPIO up.
Once the task is finished and goes to the background, it resets its GPIO down.

With the oscilloscope, we can measure the voltage of the two GPIOs and compute the real context switching time from those measurements.
The figure \ref{fig:real-context-switching-time-measurement} shows the different steps of our methodology.
On this schema, the execution time of the two tasks is represented above the voltage measurements of the two GPIOs.
The context switching time happens between the execution of the two tasks and is bordered by dotted lines.

\begin{figure}[!ht]
  \centering
  \includegraphics[scale=1]{assets/real-context-switching-time-measurement.pdf}
  \caption{\label{fig:real-context-switching-time-measurement}steps in the methodology to compute the real context switching time}
\end{figure}

The steps are the following.
The first task sets up its GPIO on step A.
The GPIO is in high position few nanoseconds later on step A'.
Once the first task is over, it resets its GPIO on step B which goes in low position on step B'.
In the same way, the second task set up its GPIO on step C which goes in high position on step C' few nanoseconds later and reset it on step D which goes to low position on step D'.
The oscilloscope measures the context switching time between the step B' and the step C'.
The time for the GPIOs to rise up or down is around 10 nanoseconds and can be omitted.


\subsection{Measurements setup\label{sec:measurement-setup}}

The oscilloscope used for the measurements was the Tektronix MSO 56\cite{mso56} available at the Welcome Lab at UCLouvain.
We used two channels to measure the voltage of the two GPIOs used by the application.
The interface of the oscilloscope, shown in the figure \ref{fig:oscilloscope-interface}, allows us to directly see in real time our measurements.
A table resume the measurements displayed in the graph below while we can check the voltage of our two GPIOs in the bottom-right window.
Once we have retrieved 1000 measurements, we export our data to a flash thumb.

\begin{figure}[!ht]
    \centering
    \includegraphics[scale=0.25]{assets/oscilloscope-interface.png}
    \caption{interface of the oscilloscope\label{fig:oscilloscope-interface}}
\end{figure}

We updated our simple task by adding GPIO calls in order for the oscilloscope to detect it.
The task will set up a GPIO, wait for 1 ms, and then clear the GPIO.
The two tasks use different GPIO in order to differentiate them with the oscilloscope.

\subsubsection{Implementation in Contiki}
The source code of the updated task is shown in the listing \ref{lst:gpio-task-code}.
\texttt{GPIO\_SET\_PIN()} and \texttt{GPIO\_CLR\_PIN()} are used to respectively set and clear the GPIO used by the task.

\begin{lstlisting}[style=CStyle, float, label={lst:gpio-task-code}, caption={source code of the task with GPIO calls}]
  PROCESS_THREAD(task, ev, data)
  {
      PROCESS_BEGIN();
  
      while (1)
      {
          GPIO_SET_PIN(
              GPIO_PORT_TO_BASE(GPIO_C_NUM), 
              GPIO_PIN_MASK(3));

          clock_delay_usec(1000);

          GPIO_CLR_PIN(
              GPIO_PORT_TO_BASE(GPIO_C_NUM), 
              GPIO_PIN_MASK(3));

          PROCESS_PAUSE();
      }
  
      PROCESS_END();
  }
\end{lstlisting}

\subsubsection{Implementation in RIOT}
The source code of the updated task is shown in the listing \ref{lst:gpio-task-code-riot}.
\texttt{gpio\_set()} and \texttt{gpio\_clear()} are used to respectively set and clear the GPIO used by the task.

\begin{lstlisting}[style=CStyle, float, label={lst:gpio-task-code-riot}, caption={source code of a task implemented in RIOT for the simple application}]
    void *thread(void *arg)
    {
        (void)arg;
    
        while (1)
        {
            gpio_set(GPIO_PIN(PORT_C, 2));

            for(int i = 0; i < 1000; i++) {}

            gpio_clear(GPIO_PIN(PORT_C, 2));

            thread_yield();
        }
        return NULL;
    }
\end{lstlisting}

\subsubsection{Boards used}

To perform our measurements, we have used two devices from Zolertia.
The RE-Mote board\cite{zolertia-remote} and the Z1 board\cite{zolertia-z1}.
The CPU of the RE-Mote is a ARM Cortex-M3\cite{arm-cortex-m3} 32 MHz clock speed with 512 kB flash and 32 kB RAM.
The Z1 have a MSP430F2617\cite{msp430} 16 MHz clock speed with 92 kB flash and a 8 kB RAM.

With those specifications, the Zolertia RE-Mote categorizes itself as a Class-2 device and the Zolertia Z1 as a Class-1 device.
Moreover, both Contiki and RIOT support those boards.

\begin{figure}[!ht]
    \begin{minipage}{.45\textwidth}
        \centering
        \includegraphics[scale=.5]{assets/remote.png}
        \caption{Zolerta RE-Mote board}
    \end{minipage}\hfill
    \begin{minipage}{.45\textwidth}        
        \centering
        \includegraphics[scale=2.5]{assets/z1.png}
        \caption{Zolerta Z1 board}
    \end{minipage}
\end{figure}

\subsection{Measurements results\label{sec:ref-measurements}}

In the end, we have four different reference values.
One for each board and for each RTOS.
The results are divided by boards and the measurements from Contiki are compared side-by-side with the measurements from RIOT for more clarity.

\subsubsection{RE-Mote measurements}
With the RE-Mote board, we have measured an average context switching time of 18.505 $\mu$s for Contiki and  12.626 $\mu$s for RIOT.
The table \ref{tab:reference-value-remote} shows the statistics of the measurements.

\begin{table}[!ht]
  \centering
  \begin{tabular}{l|c|c}
       & Contiki & RIOT OS \\ \hline
  Mean ($\mu$s) & 18.505 & 12.626 \\
  Min  ($\mu$s) & 14.312 & 12.549 \\
  Max  ($\mu$s) & 72.753   & 12.656
  \end{tabular}
  \caption{Reference value for Contiki and RIOT OS on the RE-Mote}
  \label{tab:reference-value-remote}
  \end{table}

The figure \ref{fig:reference-value-contiki-remote} and the figure \ref{fig:reference-value-riot-remote} show the distribution of the reference value with the RE-Mote board.
Note that the y-axis is logarithmic.
For the reference value of Contiki, we can see that the majority of the measurements is around 14 $\mu$s.
Some measurements are above 35 $\mu$s with extrema around 70 $\mu$s.
For RIOT, we find two distributions.
One around 12.56 $\mu$s and the other around 12.64 $\mu$s.

\begin{figure}[!ht]
    % \begin{minipage}{.45\textwidth}
        \centering
        \includegraphics[scale=.7]{assets/reference-value-contiki-remote.pdf}
        \caption{reference measurements distribution with Contiki on the RE-Mote board\label{fig:reference-value-contiki-remote}}
    % \end{minipage}\hfill
    % \begin{minipage}{.45\textwidth}        
        \centering
        \includegraphics[scale=.7]{assets/reference-value-riot-remote.pdf}
        \caption{reference measurements distribution with RIOT on the RE-Mote board\label{fig:reference-value-riot-remote}}
    % \end{minipage}
\end{figure}

\subsubsection{Z1 measurements}
With the Z1 board, we have measured a higher context switching time than with the RE-Mote board. The average context switching time for Contiki is 54.99 $\mu$s and 30.6971 $\mu$s for RIOT.
It is not a surprise that the RE-Mote board have a smaller context switching time than the Z1 board as the latter is a Class-1 device while the former is a Class-2 device.
The table \ref{tab:reference-value-z1} shows the statistics of the measurements made with the Z1 board.

\begin{table}[!ht]
  \centering
  \begin{tabular}{l|c|c}
                & Contiki & RIOT \\ \hline
  Mean ($\mu$s) & 55.077   & 27.699 \\
  Min  ($\mu$s) & 33.061 & 27.580 \\
  Max  ($\mu$s) & 547.95   & 27.827
  \end{tabular}
  \caption{Reference value for Contiki and RIOT on the Z1}
  \label{tab:reference-value-z1}
  \end{table}

The figure \ref{fig:reference-value-contiki-z1} and the figure \ref{fig:reference-value-riot-z1} show the distribution of the reference value with the Z1 board.
Note that the y-axis is logarithmic.
This time, RIOT has a strong distribution around 27.625 $\mu$s.
For Contiki, on the other hand, we have the majority of our measurements around 32 $\mu$s but some measurements are above 100 $\mu$s.

\begin{figure}[!ht]
    % \begin{minipage}{.45\textwidth}
        \centering
        \includegraphics[scale=.7]{assets/reference-value-contiki-z1.pdf}
        \caption{reference measurements distribution with Contiki on the Z1 board\label{fig:reference-value-contiki-z1}}
    % \end{minipage}\hfill
    % \begin{minipage}{.45\textwidth}        
        \centering
        \includegraphics[scale=.7]{assets/reference-value-riot-z1.pdf}
        \caption{reference measurements distribution with RIOT on the Z1 board\label{fig:reference-value-riot-z1}}
    % \end{minipage}
\end{figure}

\subsubsection{Resume}
The table \ref{tab:reference-values-resume} resume our reference values with the two boards, the RE-Mote and the Z1, and the two RTOS, Contiki and RIOT.

\begin{table}[!ht]
  \centering
  \begin{tabular}{l|c|c}
                   & Contiki & RIOT    \\ \hline
  Z1 ($\mu$s)      & 54.99   & 30.6971 \\
  RE-Mote ($\mu$s) & 19.0329 & 13.9829
  \end{tabular}
  \caption{Resume of the reference values}
  \label{tab:reference-values-resume}
  \end{table}

Those values will be used during our experiments as the real context switching time on the two boards and with the two RTOS.
To assess the performance of our benchmarking framework, we will compare the framework measurements with the reference measurements.