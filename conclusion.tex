\chapter*{Conclusion and possible improvements}

\section*{First objective}
The first objective of the thesis was to summarize the theory underlying RTOS.
We gathered what we consider the important features in a RTOS.
In some points they are similar to general purpose OS but they often implement specificities that are not found in a regular OS.
We presented 3 different embedded OS and compared them from a theoretical point-of-view.

\section*{Second Objective}
The second objective was to develop a proof-of-concept showing to implementing a framework to benchmark RTOS applications is possible.
We had two questions regarding this framework.
The first question was to know if it is possible to build a benchmarking framework capable of performing precise measurements without the use of an oscilloscope.
The second one was to determine if such a framework could be able to retrieve information usefull in the context of a RTOS.

We started by developping a kernel-integrated framework but it proved more difficult than expected.
This approach was not explored further.

The second approach was to develop a middleware between the RTOS and the application.
This method requires modifications in the application for it to call the framework.
This approach caused a large overhead due to the serial port.
Moreover, it was not able to perform precise measurements.

The third approach was developped with the issues from the second approach in mind.
With the help from the PSLab device, which serves as a monitoring device.
The PSLab benchmarks the board using GPIO and results are collected with a computer connected to the PSLab.
This method brought the best results but an overhead is still observed.

In the end, we can answer to our first question by the affirmative.
We were able to implement a framework capable of measuring with enough precision the context switching time of an application running on a RTOS.

\subsection*{Going further with the framework}

For our second question, improving the framework is possible.
With the extension approach, it is possible to store more data in a cache while the application runs.
Such information could be the memory usage of the application but also some statistics of the CPU utilization.
It would possible to determine which task runs the most or which interrupt is called the most.
Those datas stored in the cache would be written on the serial port at appropriate times.

Moreover, the PSLab is a customizable tool that can be used to measure the power consumption of the boards.