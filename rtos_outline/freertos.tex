\section{FreeRTOS}

\subsection{Historic}
%https://www.freertos.org/RTOS.html
The FreeRTOS kernel has been developped in 2003 by \href{https://www.linkedin.com/in/richard-barry-4562262/}{Richard Barry}. %fix href pls
He founded a company called Real Time Engineers Ltd to develop and maintain FreeRTOS until the stewardship of the project was passed to Amazon Web Services in 2017.
Since then, we can distinguish the FreeRTOS kernel and Amazon FreeRTOS which includes the aforementioned kernel 
    with a set of libraries extending the functionalities of the RTOS.

\subsection{Characteristics and features}
%real time
Similarly to RIOT, FreeRTOS is designed to be a real-time preemptive operating system aimed at embedded devices.
Its strength comes from the fact that it is highly and easily configurable.\\

%license
%https://www.freertos.org/a00114.html
\paragraph{License} Both the FreeRTOS kernel and Amazon FreeRTOS are provided under the MIT license\cite{FreeRTOSlicense}.
This license gives pretty much no restriction on reusability of the software.

%scheduler
%http://richardgoyette.com/Research/Papers/FreeRTOSPaper.pdf
\paragraph{Technicals} For example, FreeRTOS supports preemptive or cooperative scheduling\cite{goyette}. 
It also uses Dynamic Priority Scheduling which means that the priorities are defined during runtime.
The user can choose which scheduling policy he wants by modifying the \texttt{configUSE\_PREEMPTION} variable in the \texttt{FreeRTOSConfig.h} file.
In the cooperative case, the only action the scheduler performs is to increase the tick count.
In the preemptive case, the scheduler increments the tick count and then checks if a task is in the unblocked state.
If it is indeed the case, it checks the priority level of the task and compares it with the current task to see if a switch is required.\\

%tasks and coroutines
%https://www.freertos.org/taskandcr.html
%https://www.researchgate.net/profile/Ming_Yuan_Zhu/publication/308692183_Understanding_FreeRTOS_A_Requirement_Analysis/links/57eb301108ae5d93a4816184.pdf
Since FreeRTOS uses the multi-threading paradigm, the code can be structured as a set of tasks.
As explained in the first chapter, each task use its own stack and therefore has its own context.

In addition to tasks, FreeRTOS includes a another mechanism called \texttt{co-routine}\cite{freertos-tasks-coroutines}\cite{freertos-tasks-coroutines2}.
Co-routines are similar to tasks but have some fundamental differences which justify their usage:
\begin{itemize}
    \item They share the same stack between them which greatly decrease the memory wasted in provisioning.
    \item They use a cooperative scheduling between them but they can be used in an application using preemptive tasks.
    \item The shared stack comes with more restrictions on how co-routines can be structured compared to regular tasks.
\end{itemize}
Co-routine have been implemented for very RAM constrained devices and are very rarely used those days.
Nonetheless, they stay a relevant alternative to tasks for specific usages.\\
%ipc
%deadlock avoidance
%scheduler suspension
FreeRTOS also supports tickless idle mode, similarly to RIOT.
Since FreeRTOS is highly configurable, the developer can activate this feature 
    by setting the variable \texttt{configUSE\_TICKLESS\_IDLE} as 1 in \texttt{FreeRTOSConfig.h}.
A custom built tickless idle mode can also be provided by the developer and used by setting the same variable to 2.
This feature can be used when a default implementation in not provided for a certain board.
%memory allocation

\paragraph{Community} From a community standpoint, FreeRTOS makes the distinction between official an supported code base and user contributed code.
The first is developped by Amazon Web Services when the second is available on a distinct \href{https://interactive.freertos.org/hc/en-us}{website}.%TODO fix href
The project is very Amazon-centric and makes it difficult to federate a community.