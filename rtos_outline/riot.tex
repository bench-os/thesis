\section{RIOT}
%TODO explain why we chose riot

\subsection{Historic}
\paragraph{}
%license?
The RIOT project started privately in 2008.
It started as a part of the FeuerWhere project (\url{http://feuerwhere.mi.fu-berlin.de}), 
    where firefighters would be tracked with embedded devices during an intervention.
The goal was to design an ad-hoc self configurating network of sensors used 
    to monitor vital state and environement parameters of rescue forces inside a building.

% https://ieeexplore.ieee.org/stamp/stamp.jsp?tp=&arnumber=6142316
In 2010, a fork from the FeuerWare software developped for the FeuerWhere project was made.
Developpement continued for $\mu$kleos, a microkernel based operating system for embedded devices.
The focus of this system was modularity and Internet compliance with the integration of IETF protocols such as 6LoWPAN, RPL and TCP.

In 2013, RIOT went public.
$\mu$kleos was rebranded to avoid problems with spelling and pronounciation.
Since then, more than 200 people contributed to the project with more than 20 000 commits and 75 releases.
The project is currently supported by Freie Universität Berlin, INRIA and Hamburg University of Applied Sciences.

\subsection{Characteristics and features}
%ce que fait l'os
%https://www.osrtos.com/rtos/riot/
\paragraph{}
RIOT is an open-source real-time operating system.
The OS provides a lightweight operating system with multi-threading and real-time features running across a wide range of resource constrained devices.
It runs on 8-bit, 16-bit and 32-bit platforms and only needs a minimum of 1,5 kB of RAM.
A focus has been made on making RIOT standard-C and C++ compliant.

The RIOT project aims to bridge the gap between operating systems for Wireless Sensor Networks (WSN) and traditional fully-fledge operating systems.
RIOT follows some design objectives such as:
\begin{itemize}
    \item Energy efficiency
    \item Small memory footprint
    \item Modularity
    \item Uniform API, platform agnostic
\end{itemize}

\subsection{Specificities}
\paragraph{}
%https://hal.inria.fr/hal-00768685v1/document
%static memory for the kernel
%dynamic memory
RIOT enforces constant time for kernel tasks.
The kernel exclusively uses static memory which garantees runtimes of \textit{O}(1) whereas dynamic memory is used for applications.

The scheduler cycle is also performed in constant runtime thanks to a circular linked-list of threads.

%tickless
The main specificity of RIOT resides in its scheduler.
The operating system includes a tickless scheduler.
When there is no pending task, the operating system will switch to the \textit{idle} task which is the thread with the lowest priority.
The CPU will then put itself in its deepest possible sleep mode (which depends on the device in use).
In this state, the device can only be waken up by an interrupt (external or kernel-generated).