\section{Integration in the kernel \label{sec:kernel}}

Our first approach was to implement the framework inside the kernel of RIOT OS.
Ideally, inside the kernel source code, we could measure the time that the scheduler takes to switch from a task to an other and we have our context switching time.

This approach have some advantages.
First, it make the framework completely hidden for the developer.
This is great because it match one of our criteria for the framework.
The developer could implement its application regardless of the framework operation.
Then, one could use the framework on previously implemented application without changing the source code.

However, we choose to abandon the approach of integrating the framework inside the kernel of RIOT OS for the following reasons.
First, we are not sure that if we use this methodology we will measure the real context switching time.
It is possible that some calls or functions are executed before the scheduler and that the framework will not take them into account.
Finally, the scheduler implementation is strongly platform-dependent meaning that every platform have its own scheduler source code.
It is impossible for us to integrate our framework for each existing platforms.