\section{Power management}

%to read
%https://dl.acm.org/citation.cfm?id=2333680
%https://dl.acm.org/citation.cfm?id=860179
%https://dl.acm.org/citation.cfm?id=860184
%https://ieeexplore.ieee.org/abstract/document/4054780
%http://www.es.mdh.se/pdf_publications/327.pdf
%https://www.sciencedirect.com/science/article/pii/S030626191630678X
%https://ieeexplore.ieee.org/abstract/document/5944309
%https://ieeexplore.ieee.org/abstract/document/8617010

%read
%https://www.sciencedirect.com/science/article/pii/S0167739X18329194

The focus on energy management is something fairly recent in the field of information technology (IT).
With the rise of the Internet of Things (IoT), advancements have been made to allow operating systems to manage power consumption more efficiently.

Numerous communication stacks focused on IoT and low energy consumption have been developed in the last decade.
%TODO cite examples

Unfortunately lesser attention has been paid to the design of energy-efficient OS for resource-constrained devices.
Traditional hardware is limited in term of power management and the progress in this field required both software and hardware to evolve.
We will present below some of the advancements made and the technologies developed in recent years\cite{power-mgmt}.


\subsection{Hardware power management}
In order to implement advanced techniques of power management, certain hardware features have been developed.
The purpose is to give more control from the software over the hardware.

\subsubsection{Clock gating}
% https://m.eet.com/media/1157354/fpmm%20-%20part%201.pdf
% https://en.wikipedia.org/wiki/Clock_gating
Clock gating is a technique consisting of turning off the clocks of unused peripherals in order to save energy.
Those peripherals enter what is called \textit{idle state} or \textit{sleep}.
The clocks are physically switched off from the circuit with the addition of a logical gate and do not consume energy until reactivation.

%power down mode? dynamic power
\subsubsection{CPU low-power modes}
The recent advancements in CPU have introduced power-saving modes.
This feature stops the CPU clocks so that it is put to sleep unless
    a scheduling event or interrupt is triggered and wakes up the CPU, with the help of a real-time clock (RTC) for example.
% How the event can be scheduled if there is no clock to check the event ?

\subsubsection{Real-time clock wake up support}
% https://www.electronics-tutorials.ws/connectivity/real-time-clocks.html
When a CPU is in sleep mode, there is two possibilities to wake it up with an on-chip RTC or by an external event.
The on-chip RTC is a low-frequency clock (usually around 32kHz) that does not drain a lot of battery life.
RTC can include alarm functions: timers that when reached, trigger the RTC to wake up the processor.

\subsubsection{Supported CPU frequencies / dynamic frequency scaling}
In modern CPU, many options are available to switch between frequency ranges depending on the resources needed.
This feature can be used to minimize power consumption when  we do not need much computational power.

\subsubsection{Adaptive voltage scaling / dynamic voltage scaling}
%https://www.eetimes.com/document.asp?doc_id=1271842#
Similarly to the CPU frequency, voltage can be regulated based on the actual state of the chip.
The voltage is continuously monitored and adjusted during the runtime.


\subsection{Operating system power management}

\subsubsection{Peripherals state control}
Peripherals state control makes use of the clock-gating feature provided by the hardware.
Thanks to this feature, only the peripherals clocks required by the application at a certain point in time are active.
The other clocks are gated and do not consume energy.

\subsubsection{Sleep mode}
The idea is to allow the system to switch off certain components of the mi\-cro\-pro\-ces\-sor.

The sleep mode of a microprocessor takes advantage of multiple hardware features
    such as adaptative voltage scaling, CPU low-power modes and dynamic frequency scaling.
The RTC wake up support serves to wake up the CPU when in sleep mode and no other source is active.

\subsubsection{Tick suppression}
%https://www.embedded.com/electronics-blogs/industry-comment/4414162/1/FreeRTOS-s-tick-suppression-saves-power
Tick suppression defines the principle of providing tick-less support for the scheduler.

In a regular scheduler, a periodic timer (tick interrupt) is used to track time.
This tick interrupt wakes up the CPU to perform a scheduling cycle.
Such a mechanism, even if punctual, is frequent and then depletes power in a non-negligeable way by entering and exiting sleep mode frequently\cite{freertos-tick-supp}.

In a tick-less scheduler, the tick interrupt is disabled when the idle task is running.
Stopping the tick interrupt allows the CPU to remain in a deep power-saving state 
    until either an interrupt occurs or it is time for the kernel to switch task.

%schema tick vs tickless scheduler