\section{Scheduling}

The scheduler plays an important role in the design of a(n) (RT)OS.
The performances of an OS are deeply impacted by the scheduling algorithm it uses.
As we will see further in the thesis, it can also affect the energy consumption of the device.\\
%TODO reference for where

\subsection{Base concepts}

% What is a scheduler
A scheduler is a process designed to choose which task will run at a certain time in the system.
Different strategies can be used, each one with multiple advantages and drawbacks.
Some commonly used are presented below in the Subsection \ref{subsect:sched_policies}.

% Preemptive vs non-preemptive (cooperative)
A scheduler is said to be \textit{preemptive} if tasks in the system can preempt each other.
When a higher-priority task wants to execute, the scheduler can interrupt a lower-priority task and run the higher-priority one.
On the other hand, a scheduler is said \textit{non-preemptive} or \textit{cooperative}
    if a task that has been allowed to start will execute until it is complete.

% online/offline
Another distinction we can make is to divide schedulers into \textit{online} and \textit{offline} schedulers.
% online
Online schedulers decide of the ordering of the different tasks during runtime based on various parameters (such as task priority for example).
A scheduler based on task priority is also called \textit{priority-based} scheduler.
% offline
On the contrary, offline schedulers (also known as \textit{table-driven} schedulers) perform their scheduling decisions at the start of the system.

\subsection{Scheduling policies\label{subsect:sched_policies}}

\subsubsection{First in, first out (FIFO)}
Also known as first come, first serve (FCFS), FIFO is one of the simplest scheduling algorithms.
Processes are queued into a data structure in their order of arrival.
The first process to be enqueued is the first that will be executed.

\subsubsection{Round-robin}
The round-robin scheduling algorithm divides the time allocated to each process into fixed time slices.
If a process does not terminate when his allocated time slice is expired, the scheduler switches to another task.
If a task terminates within its time slice, the scheduler simply switches to the next task.

\subsubsection{Earliest deadline first (EDF)}
Earliest deadline first scheduling algorithm dynamically assigns a priority
    to each enqueued process (based on its deadline or an estimation of it) into a priority queue data structure.
The scheduler then executes the process the closest to its deadline at each scheduling event.

\subsubsection{Fixed-priority preemptive scheduling}
Priority for each task is pre-assigned by the operating system.
The scheduler arranges the tasks by order of priority.
Higher-priority tasks can interrupt lower-priority tasks\cite{Operatin47:online}.

%\subsection{Real-Time Scheduling}
%handbook chapter 2 12
