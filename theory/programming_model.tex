\section{Programming Model}
%definition
The programming model of an RTOS can be seen as "the way to program" an application using a specific RTOS.
Different "ways to program" or \textit{paradigms} are predominant in the world of RTOS's.
In this section, we'll present the two main different paradigm used.


\subsection{Event-driven model}
In an event-driven programming model, a program generally consists of a main loop which listens for events.
Events can be generated by interrupts, sensors or user input.
When an event is detected, a callback function is triggered.

The developper has to manually maintain state across tasks which can be tedious.
Thus, the individual tasks do not have to maintain their own stack and they use a shared-stack.
The memory footprint is then reduced since a single stack is used across the application.

\subsection{Multithreaded model}
The multithreaded model allows an application to run different tasks in their own thread context,
    and communicate between them using an Inter-Process Communication API.

Each thread has its own memory stack and does not require manual management by the programmer.
The stack is managed automatically by the thread scheduler.
The memory requirements of threaded application are often larger than their event-driven counterparts.