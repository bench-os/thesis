\chapter{Results}

% The results are split in two parts.
% The first part is the context switching time value measured by our benchmarking framework.
% For the second part, we have measured again the real context switching time with the oscilloscope the same way as for the reference measurement.

% \section{Internal benchmarking framework results}
% The value measured by our internal framework is represented in the table \ref{tab:internal-framework-measurement}.

% \begin{table}[!ht]
%   \centering
%   \begin{tabular}{llll}
  & \multicolumn{3}{c}{Time ($\mu$s)}          \\ \cline{2-4} 
  & \multicolumn{1}{c}{Mean} & Min  & Max  \\ \cline{2-4} 
From task 1 to task 2 & 7812                     & 7812 & 7812 \\
From task 2 to task 1 & 7812                     & 7812 & 7812
\end{tabular}
%   \caption{Context switching time measured by our internal benchmarking framework}
%   \label{tab:internal-framework-measurement}
%   \end{table}

% \section{External benchmarking framework results}
% The context switching time computed by our external framework is displayed in the table \ref{tab:external-framework-measurement}.

% \begin{table}[!ht]
%   \centering
%   \begin{tabular}{llll}
  & \multicolumn{3}{c}{Time ($\mu$s)}                             \\ \cline{2-4} 
  & \multicolumn{1}{c}{Mean} & Min  & \multicolumn{1}{c}{Max} \\ \cline{2-4} 
Context switching time & 18.93                     & 15.62 & 40.26                    \\
\end{tabular}
%   \caption{Context switching time measured by our external benchmarking framework}
%   \label{tab:external-framework-measurement}
% \end{table}

% \section{Oscilloscope results}
% Using the same setup as for the reference measurement, we have measured again the real context switching time with both our internal and external benchmarking framework.

% The figure \ref{fig:internal-framework-value-wave} shows the voltage measurement of the two GPIOs used by the tasks while using our internal framework.
% The figure \ref{fig:external-framework-value-wave} shows the voltage measurement of the single GPIO used by the external framework while using it.

% \begin{figure}[!ht]
%   \centering

%   \begin{minipage}{0.45\textwidth}
%     \includegraphics[width=0.9\textwidth]{assets/framework-value-wave.png}
%     \caption{\label{fig:internal-framework-value-wave}Internal benchmarking framework voltage measurement}

%   \end{minipage}\hfill
%   \begin{minipage}{0.45\textwidth}

%     \includegraphics[width=0.9\textwidth]{assets/external-framework-value-wave.png}
%     \caption{\label{fig:external-framework-value-wave}External benchmarking framework voltage measurement}

%   \end{minipage}
% \end{figure}

% The table \ref{tab:frameworks-oscilloscope-comparison} shows a comparison of the real context switching time measured with the oscilloscope between the internal framework and the external one.

% \begin{table}[!ht]
%   \centering
%   \begin{tabular}{llllllll}
  & \multicolumn{7}{c}{Time ($\mu$s)}                                                      \\ \cline{2-8} 
  & \multicolumn{3}{c}{Internal framework} &  & \multicolumn{3}{c}{External framework} \\ \cline{2-4} \cline{6-8} 
  & \multicolumn{1}{c}{Mean} & Min  & Max  &  & Mean        & Min         & Max        \\ \cline{2-4} \cline{6-8} 
Context switching time & 1865                     & 1862 & 1866 &  & 14.87       & 14.79       & 14.97      \\
Duration of task 1     & 1003                     & 1003 & 1003 &  & 1003        & 1003        & 1003       \\
Duration of task 2     & 1003                     & 1003 & 1003 &  & 1003        & 1003        & 1003      
\end{tabular}
%   \caption{Context switching times and task durations measured with the oscilloscope using our internal and external benchmarking frameworks}
%   \label{tab:frameworks-oscilloscope-comparison}
% \end{table}

% \section{Discussions}

% \subsection{Internal benchmarking framework}

% By comparing with our reference measurement, the first assessment we can make is that our benchmarking framework does not compute the context switching time correctly.
% The framework measure a context switching time of 7812 $\mu$s where we expect a value of 14.68 $\mu$s.

% Second assessment we can make is that our framework add a huge overhead.
% When comparing the figure \ref{fig:measurement-value-wave} with the figure \ref{fig:internal-framework-value-wave}, the overhead is largely visible.
% By comparing the values measured by the oscilloscope, the reference measurement and the real context switching time while using our framework, we see that our framework add an overhead of 1850 $\mu$s.
% The table \ref{tab:measurements-comparison} shows this comparison.

% \begin{table}[!ht]
%   \centering
%   \begin{tabular}{lll}
%                         & \multicolumn{2}{c}{Time ($\mu$s)}                                     \\ \cline{2-3} 
%                         & \multicolumn{1}{c}{No framework} & Framework \\ \cline{2-3} 
%   From task 1 to task 2 & 14.68                                     & 1864                  \\
%   From task 2 to task 1 & 14.88                                     & 1865                  \\
%   Duration of task 1    & 1003                                      & 1003                  \\
%   Duration of task 2    & 1003                                      & 1003                 
%   \end{tabular}
%   \caption{Comparison of the oscilloscope measurements}
%   \label{tab:measurements-comparison}
% \end{table}

% \subsubsection{Limitations of the framework}
% As the results show, our framework suffers from several limitations.

% First, the devices can not compute the context switching time with enough precision.
% This lack of precision is due of the speed of the device clock that is not high enough.
% In our experiment, the Zolertia RE-MOTE used with Contiki could only use a timer with 128 ticks per second or 1 tick every 7812 $\mu$s.

% Second, our framework output its computed value through serial port.
% Printing out at least 32 characters for every context switch on the serial port even at 250 kbit/s took 1.2ms.
% One optimization could be to reduce the number of bits send or use a cache.

% \subsection{External benchmarking framework}

% The first thing we can see is that our external benchmarking framework is much more precise that our internal benchmarking framework.
% Even if the result does not match our excpectation, the results show an improvement.
% We went from a measurement error with a factor of more than 500 to a measurement error with a factor less than 2.

% Secondly, with the real context switching time measured at 14.87$\mu$s while our framework was on, our external benchmarking framework does not add an overhead.

% In conclusion, our external benchmarking framework is more precise than our internal benchmarking framework and does not add an overhead to the real context switching time.